\input texinfo
@settitle 3DViewer v2.1 Manual

@copying
This manual is for 3DViewer v2.1 project.

Copyright @copyright{} 2022 bperegri irobin
@end copying

@titlepage
@title 3DViewer v2.1 Manual
@author by bperegri, irobin
@page
@vskip 0pt plus 1filll
@insertcopying
@end titlepage

@contents

@node Top

@menu
* Loading a model
* Visualisation area
* Affine transformations
* Model information
* Settings and Conditions
* Record
@end menu

@node Loading a model
@chapter Loading a model

You can load a model by clicking a button "Open Model" (1 in upper toolbar) and chosing an .obj file.

The model will be displayed in visualisation area.

The path to currently loaded file will be displayed in Information window, accessible through clicking on "info about model" button (5 in upper toolbar).

You can also load a texture by clicking "Texture" button (6 in upper toolbar). Click it again to disable it.

You can remove a model from visualisation area with "delete Models" button (11 in upper toolbar).

@node Visualisation area
@chapter Visualisation area

Visualisation area contains visualisation of a loaded model.

You can change the view angle and distance to the model by clicking and dragging mouse over the visualisation area and by turning the mouse wheel.

You can change display method in the menu appearing on clicking "Settings Model" button (3 in lower toolbar).

You can change colors of a model and bacground in menu appearing on clicking "Colors" button (2 in lower toolbar).

You can turn on/off light by clicking "on/off light" button (7 in upper toolbar), and change light source settings in menu appearing on clicking "Settings Light" button (4 in lower toolbar).

You can switch between flat shading and soft shading modes by clicking "Shadding" button (8 in upper toolbar).

@node Affine transformations
@chapter Affine transformations

3DViewer v2.0 allows to perform affine transformations on a model.

It can be done in menu appearing on clicking "Model Matrix" button (1 in lower toolbar).

Input fields can be filled directly, using keyboard, or via up/down button included in the input field.

To revert all transformations click button with two arrows in this menu.

You can also adjust transformation limits in submenu.

"Axis Rotate" button (10 in upper toolbar) opens a menu where you can start and stop rotation of a model around itself in various vays, with different speed and direction.

@node Model information
@chapter Model information

You can see various stats and information entries in a dialog window by clicking "info about model" button (5 in upper toolbar).

@node Settings and Conditions
@chapter Settings and Conditions

Click the "Settings Screen" button (5 in lower toolbar), to open a menu to chose main window sizes. You can also change them by dragging lower right corner of the window.

You can switch parallel and central projections by clicking "Projection" button (9 in upper toolbar).

You can open a menu to save conditions of a model (position, colors, texture) by clicking "Back Up" button (6 in lower toolbar).

@node Record
@chapter Record

"screenShot" button (2 in upper toolbar) creates screenshots of visualisation area in .bmp and .jpeg formats, saving them in screenshots folder. Names of saved files will be shown in status bar at the bottom of the window.

"make Gif" button (3 in upper toolbar) creates a gif-animation by recording affine transformations and other changes you perform in real time. Recording progress will be shown as a progress bar in lower right corner. Name of a saved file will be shown in status bar at the bottom of the window.

"GIF 360" button (4 in upper toolbar) creates a gif-animation of a model making one full turn around itself.

@bye